% @Author: ArthurBernard
% @Email: arthur.bernard.92@gmail.com
% @Date: 2019-07-31 22:09:26
% @Last modified by: ArthurBernard
% @Last modified time: 2024-10-10 18:13:49
\documentclass[a4paper]{arthur-cv}
\title{Curiculum Vitae}
\author{Arthur Bernard}

\usepackage[english]{babel}
\usepackage{fontspec}
\usepackage{microtype}
\usepackage{hyperref}
\usepackage{xcolor}

% Customize colorthem
% \definecolor{leftcolorband}{HTML}{e0e0e0}
% \definecolor{boxcolor}{HTML}{851919}
% \definecolor{maincolor}{HTML}{420c0c}
% \definecolor{secondcolor}{HTML}{861919}
% \definecolor{thirdcolor}{HTML}{591111}

% Define color for hyperlink
\definecolor{colhyperlink}{HTML}{0E5484}

% Set profile info
\profilepic{pictures/qrcode_github.jpeg}
\cvname{Arthur Bernard}
\cvlinkedin{/in/arthur-bernard-789955152}
\cvgithub{ArthurBernard}
\cvmail{arthur.bernard.92@gmail.com}
\cvnumberphone{+33 6 59 29 14 50}
\cvjobtitle{\normalsize Experienced Data Scientist and Python developer, specialized in the application and customization of Large Language Model (LLM).}
\cvsite{}
\cvaddress{}
\cvyearsold{}

\begin{document}
\makeprofile % Set header

\begin{textblock}{20.5}(0.25, 3.5)

  \begin{minipage}[t]{0.37\textwidth}

    %%%%%%%%%%%%%%%%%
    %%  Left side  %%
    %%%%%%%%%%%%%%%%%

    \sectionleft{Key skills}
      \subsectionleft{Retraining LLM on specifics datasets.}{}

      \subsectionleft{Expertise in Retrieval-Augment Generation (RAG) techniques:}{contextual enhancement.}

      \subsectionleft{Proficiency in API integration for complex LLM applications.}{}

      \subsectionleft{Development of user-friendly graphical interfaces for LLM applications.}

      \subsectionleft{Advenced data collection and cleaning methodologies for LLM training.}

    \sectionleft{Programming}
      \subsectionleft{Highly advanced:}{\textbf{Python} (expertise in NumPy, Pandas, Cython, PyTorch, Sickit-Learn, Flask, StreamLit, Transformers, HuggingFace, etc).}

      \subsectionleft{Advanced:}{\textbf{Shell}, VBA, \LaTeX.}

      \subsectionleft{Moderate:}{\textbf{C++}, HTML\&CSS and JS.}
  
    \sectionleft{MOOCs}
      \subsectionleft{2023 - \href{https://coursera.org/share/384d3a778f22d1a0f5fcb75338c4052d}{\textbf{Sequence Models}} (Natural Language Processing, Recurent Neural Network, Attention Models and Transformers),}{on Coursera}

      \subsectionleft{2018 - \textbf{Machine Learning}, Stanford University course,}{on Coursera.}

      \subsectionleft{And other diverse courses about \href{https://openclassrooms.com/fr/course-certificates/7660645971}{\textbf{Python}}, \textbf{Linux}, \textbf{C++}, \textbf{HTML\&CSS}, \textbf{JavaScript}, \textbf{Git}, etc.}{}

    \sectionleft{Interests}
      \subsectionleft{Artificial intelligence:}{Natural Language Processing (NLP) and time-series.}

      \subsectionleft{Crypto-currencies/Blockchains.}{}

      \subsectionleft{Various Open Source projects.}{}

  \end{minipage}\hfill\begin{minipage}[t]{0.61\textwidth}

    %%%%%%%%%%%%%%%%%%
    %%  Right side  %%
    %%%%%%%%%%%%%%%%%%
  
    \section{Experiences}
      \begin{rightenv}
        % \subsectionright{Dec. 2022 – Present}{Freelance Data Scientist}[at \textbf{\href{https://llm-solutions.fr/}{LLM Solutions}}][]{Development of an innovative solution enabling companies to \textbf{internally integrate a customized Large Language Model (LLM)}, including \textbf{model retraining} and the use of \textbf{Retrieval-Augmented Generation (RAG)} techniques on their private data.\\\textbf{Technologies Used:} Python, Flask, Streamlit, GPU Computing, RAG Techniques, LLM Retraining, etc.}
        \subsectionright{Dec. 2022 – Present}{Freelance Data Scientist}[at \textbf{\href{https://llm-solutions.fr/}{LLM Solutions}}][]{Operating as a freelance data scientist under the name LLM Solutions, offering AI customization services to businesses. Developed innovative solutions enabling companies to \textbf{internally integrate customized LLMs}, including \textbf{model retraining} and the use of \textbf{Retrieval-Augmented Generation (RAG)} techniques on their private data.\\\textbf{Technologies Used:} Python, Flask, Streamlit, GPU Computing, RAG Techniques, LLM Retraining, etc.}

        \subsectionright{Mar. 2020 – Sep. 2021}{Data Scientist Consultant}[at \textbf{Coperneec}][Paris]{For \textbf{Natixis} in Risk Management Department in the Market Stress Test team: automation of recurrent tasks, R\&D on reverse stress test methodology, \textbf{fast pricing of products}, python package to manage data across different internal data sources (\textbf{SQL}, \textbf{MDX}, \textbf{Hadoop}), etc.}

        \subsectionright{Jun. 2018 – Oct. 2019}{Quant Researcher}[at \textbf{Napoleon Group}][Paris]{R\&D of trading strategies, \textbf{portfolio allocation} algorithms, \textbf{multivariate prediction} with neural networks, \textbf{execution order algorithms}, development of \textbf{backtesting} and \textbf{financial analysis} tools, and webscraping data.}
      \end{rightenv}

    \section{Personal projects}
      \begin{rightenv}
        \subsectionright{2023 – 2024}{
          Customized LLM: \href{http://hackernews-gpt.com}{\textcolor{colhyperlink}{\textit{HackerNewsGPT}}} project
        }{\textbf{Objective:} Developed solution to enhance technology monitoring by enabling users to ask real-time questions about the latest articles from \href{https://news.ycombinator.com/}{\textcolor{colhyperlink}{\textit{HackerNews}}}.\\\textbf{Role \& Contributions:} Retraining a LLM using RAG to incorporate the latest articles for real-time, context-aware question answering. Managed the backend on a GPU server with a Flask API, and developed a user-friendly web app interface using Streamlit.}

        \subsectionright{2018 – 2019}{Machine/deep learning tools adapted to finance}{Development of a Python and Cython package to create \textbf{neural networks}, \textbf{backtest strategies}, analysis with \textbf{econometric models} and \textbf{financial indicators}, etc. Published on PyPI as \href{https://github.com/ArthurBernard/Fynance}{\textcolor{colhyperlink}{\textit{fynance}}}.}

        \subsectionright{2016 – 2020}{Trading bot algorithms on crypto-currencies}{Development and maintenance of trading bots with Python and Bash scripts. Starting in 2016 with \textbf{arbitrage strategy}, and more recently create \textbf{strategies with neural network}. Partly available on my GitHub in the repository \href{https://github.com/ArthurBernard/Trading_Bot}{\textcolor{colhyperlink}{\textit{Trading\_Bot}}}.}
      \end{rightenv}

    \section{Education}
      \begin{rightenv}
        \subsectionright{2017 – 2018}[Master's Degree]{Econometrics of Banking and Financial markets}[at \textbf{Aix-Marseille School of Economics}][Marseille]{\textbf{Courses:} Stochastic finance, financial econometrics, financial engineering, econometrics of exchange rates, neural network, etc.}%\\\textbf{Projects:} Intraday analysis of BTCUSD versus EURUSD, etc.\\\textbf{Master thesis:} Analysis of dynamics of Bitcoin.}

        \subsectionright{2013 – 2016}[Bachelor's degree]{Economics and Management, specialized in Finance}[at \textbf{Aix-Marseille University}][Marseille]{\textbf{Courses:} Time series econometrics, financial markets, statistics, optimization, informatics (SQL and VBA), etc.}
      \end{rightenv}

    \section{Miscellaneous}
      \begin{rightenv}
        \subsectionright{2019}{Data-science competition}[at \textbf{ENS Challenge Data}]{\href{http://datachallenge.cfm.fr/t/end-of-year-ranking-2019-official-top-10/243}{\textcolor{colhyperlink}{$6^{th}$} place} out of $100$+ competitors in a daily US stock market prediction challenge, organized by \textbf{Capital Fund Management}.}

        \subsectionright{Present}{Hobbies}{Cooking, sports (swimming, cycling and running), and DIY projects.}
      \end{rightenv}

  \end{minipage}

\end{textblock}

\end{document}
