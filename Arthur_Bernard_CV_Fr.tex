% @Author: ArthurBernard
% @Email: arthur.bernard.92@gmail.com
% @Date: 2019-07-11 00:05:23
% @Last modified by: ArthurBernard
% @Last modified time: 2019-07-11 23:41:36
\documentclass[a4paper]{arthur-cv-fr}
\title{Curiculum Vitae}
\author{Arthur Bernard}

\usepackage[francais]{babel}
\usepackage{fontspec}
\usepackage{microtype}

% Set profile info
\profilepic{photo_arthur_bernard.jpeg}
\cvname{Arthur Bernard}
\cvlinkedin{/in/arthur-bernard-789955152}
\cvgithub{ArthurBernard}
\cvmail{arthur.bernard.92@gmail.com}
\cvnumberphone{+33 6 59 29 14 50}
\cvaddress{7, Rue des Innocents, Paris, 75001}
\cvjobtitle{Data Scientiste en Finance Quantitative}
\cvsite{}
\cvage{26 ans}

\begin{document}
\makeprofile % Set header

\begin{textblock}{20.5}(0.25, 3.5)

  \begin{minipage}[t]{0.37\textwidth}

    %%%%%%%%%%%%%%%%%
    %%  Left side  %%
    %%%%%%%%%%%%%%%%%

    \sectionleft{Compétences}
      \subsectionleft{Maîtrise des logiciels de \textbf{statistiques} et d'\textbf{économetrie}:}{R (avancé), Octave/Matlab (avancé), SAS (débutant), STATA (débutant).}
      \subsectionleft{Systèmes d'exploitations:}{Windows et \textbf{Unix}.}
      \subsectionleft{Langues:}{Français (\textbf{langue maternelle}), Anglais (\textbf{compétences professionnels}, langue d'enseignement du Master 1 \& 2).}

    \sectionleft{Programmation}
      \subsectionleft{Très avancé:}{\textbf{Python} (expertise avec NumPy, Pandas, Cython, PyTorch, Keras, Sickit-Learn, Asyncio, Multi-process/thread, etc).}
      \subsectionleft{Avancé:}{\textbf{Shell}, VBA, LaTeX.} %Matlab/Octave,
      \subsectionleft{Débutant:}{\textbf{C++}.}
      % \subsectionleft{Basic knowledge:}{R, SAS.}
  
    \sectionleft{MOOC's}
      \subsectionleft{Apprendre à programmer avec \textbf{Python},}{OpenClassRooms.}
      \subsectionleft{\textbf{Machine Learning}, par Andrew Ng,}{Coursera.}
      \subsectionleft{\textbf{Deep Learning}, par Andrew Ng,}{Coursera.}
      \subsectionleft{Plus divers cours (\textbf{Linux}, \textbf{C++}, etc.).}{}

    \sectionleft{Centres d'intérêts}
      \subsectionleft{Intelligence artificielle.}{}
      \subsectionleft{Crypto-monnaies/Blockchains.}{}

  \end{minipage}\hfill\begin{minipage}[t]{0.61\textwidth}

    %%%%%%%%%%%%%%%%%%
    %%  Right side  %%
    %%%%%%%%%%%%%%%%%%
  
    \section{Expériences professionnelles}
      \begin{rightenv}
        \subsectionright{Janv. 2019 – Présent}{Recherche en Finance Quantitative}[Napoleon Crypto][Paris]{R\&D en stratégie de trading, \textbf{prédiction mulit-variés} avec des réseaux de neuronnes, \textbf{algorithmes d'exécution d'ordres}, développement d'outils de \textbf{backtesting}, \textbf{d'analyse financière} et de webscraping de données.}
        \subsectionright{Juin 2018 – Déc. 2018}{Stagiaire en Finance Quantitative}[Napoleon Crypto][Paris]{Recherche de stratégies quantitatives et algorithmes d'\textbf{allocation de portefeuille}. Elaboration d'une \textbf{compétition de sata-science} pour le Collège de France.}
        \subsectionright{Sept. 2013 – Mai 2018}{Administrateur (bénévole)}[Mutuelle des Etudiants de Provence]{Approbation des budgets et de la politique interne.}
      \end{rightenv}

    \section{Projets personnels}
      \begin{rightenv}
        \subsectionright{2018 – 2019}{Outils de machine/deep learning adapté à la finance}{Développement d'un package en python et cython pour créer des \textbf{réseaux de neuronnes}, \textbf{backtester des stratégies}, analyser des \textbf{models éconmetric} et faire \textbf{indicateurs financiers}, etc. Publié sur PyPI sous le nom \textit{fynance}.}
        \subsectionright{2017 – 2018}{Package de webscraping}{Développement d'un package python pour \textbf{télécharger} et \textbf{mettre à jour une base de données} sur les crypto-monnaies. Publié sur PyPI sous le nom \textit{dccd}.}
        \subsectionright{2016 – 2019}{Algorithmes de trading sur les crypto-monnaies}{Développement et maintenance de bots de trading avec des scripts en python et bash. Commencé en 2016 avec des \textbf{stratégies d'arbitrages}, et plus récemment creation de \textbf{stratégies avec des réseaux de neuronnes}. Une partie des scripts est disponnible sur mon GitHub dans le répertoire \textit{Strategy\_Manager}.}
      \end{rightenv}

    \section{Education}
      \begin{rightenv}
        \subsectionright{2017 – 2018}[Master 1 \& 2]{Econometric of Banking and Financial markets}[Aix-Marseille School of Economics][Marseille]{\textbf{Cours :} Stochastic finance, financial econometrics, financial engineering, econometrics of exchange rates, neural network, etc.\\\textbf{Projets :} Intraday analysis of BTCUSD versus EURUSD, etc.\\\textbf{Mémoire :} Analysis of dynamics of Bitcoin.}
        \subsectionright{2013 – 2016}[Licence]{Economie et Gestion}[Aix-Marseille Université][Marseille]{\textbf{Spécialisation:} Finance.\\\textbf{Cours :} Econométrie des séries temporelles, finance de marché, statistique, optimisation, informatique (SQL et VBA), etc.}
        \subsectionright{2012}[Baccalauréat]{Science}[Lycée Marie Madeleine Fourcade][Gardanne]{}
      \end{rightenv}

    \section{Divers}
      \begin{rightenv}
        \subsectionright{2019}{Compétition de data-science}[ENS Challenge Data]{$7^{ème}$ au classement provisoir. Prédiction des mouvements quotidiens des actions du maché Américain, proposé par Capital Fund Management.}
        \subsectionright{2014 – 2016}{Fondateur et secrétaire général d'une association étudiante}{Organisation et gestion de projets en équipe.}
        \subsectionright{Présent}{Hobbies}{Cuisine, voyage (Norvège, Ecosse, Pays de l'Est, etc.), natation (compétition) et théâtre.}
      \end{rightenv}

  \end{minipage}

\end{textblock}

\end{document}
